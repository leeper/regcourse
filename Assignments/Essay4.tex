\documentclass[a4paper,11pt]{article}
\usepackage[margin=1in]{geometry}
\usepackage{hyperref}

\title{Essay 4: Generalized Linear Models}
\author{}
\date{}

\begin{document}

\maketitle

\vspace{-3em}

\noindent {\bfseries Due: 8. May at 12:00 to Thomas via Blackboard}

\vspace{1em}

\noindent The purpose of this assignment is to evaluate your ability to analyze data involving non-linear/non-continuous outcome data. Your task is as follows:
\begin{enumerate}\itemsep1em
\item We provide you with a research question, hypothesis, and outcome variable(s). Translate the hypotheses into credible regression specifications to identify the hypothesized causal relationship. You will perform this analysis using cross-sectional data on individuals. 

\item Because we are providing you with the research question, hypothesis, and outcome variables, you should \textit{not} write about these issues in your assignment. Your assignment should start with your hypothesis and the ``methods'' section of your paper, which should describe the variables you have chosen to include in your analysis, how the most important of these variables are coded, and how you specify your models (this should be approximately one page of text). You can then report your ``results''. You do \textit{not} need a conclusion.

\item Your research question is: ``What effects do features of one's work environment have on satisfaction with one's financial situation?''

\item Your hypothesis is that being a public sector employee decreases one's financial satisfaction, but that this effect of employment sector may be heterogeneous (i.e., moderated by another factor or by multiple factors which you should state in your hypothesis).

\item You will test this hypothesis using data from the 2002 General Social Survey (available on Blackboard).

\item Your outcome variable is a three-category measure of satisfaction called \texttt{satfin}. You should analyze this measure as a continuous variable, as an ordered variable, and as a binary variable. You may have to recode it for some of these specifications.

\item Use appropriate regression analyses to test your hypothesis.

\item Interpret the statistical significance and substantive size and meaning of the most theoretically important relationships using regression coefficient estimates, predicted or fitted outcome values, and/or relevant marginal effects, as appropriate. Present these in tables, figures, and text.

\end{enumerate}

\end{document}
