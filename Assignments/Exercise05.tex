\documentclass[a4paper,12pt]{article}
\usepackage[top=0in, bottom=1in, left=1in, right=1in]{geometry}

\title{Tidy Data Activity}
\author{}
\date{}

\begin{document}

\maketitle

\vspace{-2em}

\noindent Examine the data below for various countries at various points in time. Transcribe those data into a tidy, rectangular dataset that you could analyze in Stata. You may do this using Stata's data editor, Microsoft Excel, a Microsoft Word table, a text editor, or whatever other computer application you choose. Stata or Excel is probably best.

\begin{enumerate}\itemsep1em

\item The percentage of a country's population that serves in the armed forces varies considerably. In Denmark it is 0.64, in Colombia it is 1.99, in Afghanistan it is 4.37, and in North Korea it is 9.31.

\item The 2014 unemployment rate in these countries is as follows. North Korea: 4.4, Colombia: 11.6, Denmark: 7.6, United Kingdom: 7.8.

\item In 2010 those numbers were: North Korea: 4.5, Colombia: 12.0, Denmark: 7.5, United States: 9.7, Great Britain: 7.9.

\item National legislatures have increasingly high numbers of female members, but growth is slow and between-country differences are high. The following data shows increases between 2010 and 2014:

\begin{center}
\begin{tabular}{lrr}
Country & 2010 & 2014 \\ \hline
Afghanistan & 28 & 28 \\
Colombia & 13 & 12 \\
Cuba & 43 & 49 \\
Denmark & 38 & 39 \\
Korea, North & 16 & 16 \\
United Kingdom & 22 & 23 \\
United States & 17 & 18 \\
\end{tabular}
\end{center}

\item We can also categorize countries into Democracies and Dictatorships:

\begin{center}
\begin{tabular}{ll}
Democracies & Dictatorships \\ \hline
Afghanistan & Cuba \\
Britain	& North Korea \\ 
Colombia & Egypt \\ 
Denmark \\
USA \\ \hline
\end{tabular}
\end{center}

\item Denmark is a highly developed country. It scores .898 on the United Nation's Human Development Index (HDI), has 10 years of compulsory education, an HIV infection rate of 0.2\%, an infant mortality rate of 3.3 and life expectancy of 79.09 years.

\item Cuba and Afghanistan differ considerably but also have similarities. Afghanistan's HIV infection rate is 0.1, Cuba's is 0.2. Afghanistan scores 0.453 on the UN HDI, Cuba's score is 0.824. Afghanistan has 9 years of compulsory education, so does Cuba. Afghanistan's infant mortality rate is 74.5, Cuba's is 4.6. Afghanistan's life expectancy is 48.28 years, and Cuba's is 78.96.

\end{enumerate}

\end{document}