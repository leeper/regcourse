\input{preamble}

\title{Regression Results in Stata}

\date[]{February 18, 2015}

\begin{document}

\frame{\titlepage}

\section{Regression Results}
\frame{\tableofcontents[currentsection]}

\frame{
	\frametitle{How Stata Works}
	\begin{enumerate}\itemsep1em
	\item Load data into memory
	\item You tell Stata to do something
	\item If valid, Stata:
		\begin{itemize}
		\item stores some results temporarily
		\item prints some results to the console
		\end{itemize}
	\item New commands overwrite stored results
	\end{enumerate}
}

\begin{frame}[fragile]
	\frametitle{Ex.: \texttt{summ} stores values in \texttt{r()}}
	\scriptsize
	\begin{verbatim}
	Scalars   
	  r(N)           number of observations
	  r(mean)        mean
	  r(skewness)    skewness (detail only)
	  r(min)         minimum
	  r(max)         maximum
	  r(sum_w)       sum of the weights
	  r(p1)          1st percentile (detail only)
	  r(p5)          5th percentile (detail only)
	  r(p10)         10th percentile (detail only)
	  r(p25)         25th percentile (detail only)
	  r(p50)         50th percentile (detail only)
	  r(p75)         75th percentile (detail only)
	  r(p90)         90th percentile (detail only)
	  r(p95)         95th percentile (detail only)
	  r(p99)         99th percentile (detail only)
	  r(Var)         variance
	  r(kurtosis)    kurtosis (detail only)
	  r(sum)         sum of variable
	  r(sd)          standard deviation
	
	\end{verbatim}
\end{frame}

\begin{frame}[fragile]
	\frametitle{\texttt{regress} stores values in \texttt{e()}}
	\scriptsize
	\begin{verbatim}
	Scalars        
	  e(N)                number of observations
	  e(mss)              model sum of squares
	  e(df_m)             model degrees of freedom
	  e(rss)              residual sum of squares
	  e(df_r)             residual degrees of freedom
	  e(r2)               R-squared
	  e(r2_a)             adjusted R-squared
	  e(F)                F statistic
	  e(rmse)             root mean squared error
	  e(ll)               log likelihood under additional assumption of i.i.d.
	                        normal errors
	  e(ll_0)             log likelihood, constant-only model
	  e(N_clust)          number of clusters
	  e(rank)             rank of e(V)
	  
	\end{verbatim}
\end{frame}

\begin{frame}[fragile]
	\frametitle{\texttt{regress} stores values in \texttt{e()}}
	\scriptsize
	\begin{verbatim}
	Macros         
	  e(cmd)              regress
	  e(cmdline)          command as typed
	  e(depvar)           name of dependent variable
	  e(model)            ols or iv
	  e(wtype)            weight type
	  e(wexp)             weight expression
	  e(title)            title in estimation output when vce() is not ols
	  e(clustvar)         name of cluster variable
	  e(vce)              vcetype specified in vce()
	  e(vcetype)          title used to label Std. Err.
	  e(properties)       b V
	  e(estat_cmd)        program used to implement estat
	  e(predict)          program used to implement predict
	  e(marginsok)        predictions allowed by margins
	  e(asbalanced)       factor variables fvset as asbalanced
	  e(asobserved)       factor variables fvset as asobserved
	  
	\end{verbatim}
\end{frame}


\begin{frame}[fragile]
	\frametitle{\texttt{regress} stores values in \texttt{e()}}
	\scriptsize
	\begin{verbatim}
	Matrices       
	  e(b)                coefficient vector
	  e(V)                variance-covariance matrix of the estimators
	  e(V_modelbased)     model-based variance
	
	Functions      
	  e(sample)           marks estimation sample
	  
	\end{verbatim}
\end{frame}

\frame{
	\frametitle{What do we do with this?}
	\begin{alltt}
	reg growth lcon\\
	
	* store fitted values as yhat\\
    predict yhat\\
	
	* store residuals as resid\\
	predict resid, r\\
	
	* plot residuals\\
	graph twoway resid lcon
	\end{alltt}
}

\frame{
	\frametitle{What do we do with this?}
	\begin{itemize}\itemsep1em
	\item We often need to explore results
	\item Default results in Stata console are:
		\begin{itemize}
		\item Ugly
		\item Formatted badly for print
		\item Includes superfluous information
		\end{itemize}
	\item We can store results and print them
	\end{itemize}
}

\frame{
	\frametitle{What do we want to present?}
	\begin{itemize}\itemsep1em
	\item Coefficients
	\item Standard Errors
	\item Significance stars (?)
	\item Model fit
		\begin{itemize}
		\item Adjusted-$R^2$
		\item RMSE ($\hat{\sigma}$)
		\end{itemize}
	\item Model-specific sample size
	\end{itemize}
}

\begin{frame}[fragile]
	\small 
	\frametitle{Results with \texttt{estimates} I}
	\begin{verbatim}
	quietly reg growth lcon
	estimates store small
	
	* print results
	estimates table
	
	* print results with summary stats
	estimates table, se stats(N r2_a)
	
	* format printing of coefs/SEs
	estimates table small large, ///
	b(%9.2f) se(%9.2f) stats(N r2_a)
	
	\end{verbatim}
\end{frame}


\begin{frame}[fragile]
	\small 
	\frametitle{Results with \texttt{estimates} II}
	\begin{verbatim}
	quietly reg growth lcon lconsq
	estimates store large
	
	estimates table small large
	estimates table small large, se stats(N r2_a)
	
	\end{verbatim}
\end{frame}

\frame{
	\frametitle{Results with \texttt{estout}}
	\begin{itemize}\itemsep1em
	\item Stata has lots of user-defined add-on packages
	\item These can be helpful
	\item We'll use one called \texttt{estout}\footnote{\url{http://repec.org/bocode/e/estout/index.html}}
	\item Install with:\\
		\texttt{ssc install estout, replace}
	\end{itemize}
}

\frame{
	\frametitle{Using \texttt{estout}}
	\begin{itemize}\itemsep1em
	\item Two functions:\\
		\begin{itemize}
		\item \texttt{esttab}: Simple regression tables
		\item \texttt{estout}: Better formatted tables
		\end{itemize}
	\item Generally, use \texttt{esttab}
	\item Use \texttt{eststo} to store multiple models
	\item Lots of options (see \texttt{help esttab})
	\end{itemize}
}


\begin{frame}[fragile]
	\small 
	\frametitle{Results with \texttt{esttab} I}
	\begin{verbatim}
	reg growth lcon
	
	* print results (but note t-statistics by default)
	esttab
	
	* better formatting
	esttab, b(%9.2f) se(%9.2f)
	
	\end{verbatim}
\end{frame}

\begin{frame}[fragile]
	\small 
	\frametitle{Results with \texttt{esttab} II}
	\begin{verbatim}
	quietly reg growth lcon
	eststo
	
	quietly reg growth lcon lconsq
	eststo
	
	estout, b(%9.2f) se(%9.2f) stats(rmse)
	esttab, b(%9.2f) se(%9.2f) stats(rmse)
	
	\end{verbatim}
\end{frame}

\begin{frame}[fragile]
	\small 
	\frametitle{Results with \texttt{esttab} III}
	\begin{verbatim}
	* output results to RTF (for MS Word)
	esttab using example.rtf
	
	* Stata won't overwrite a file
	* But you can `replace` or `append`:
	
	esttab using example.rtf, replace
	
	esttab using example.rtf, append
	
	\end{verbatim}
\end{frame}



\appendix
\frame{}

\end{document}
