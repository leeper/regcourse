\input{preamble}

\title{PhD Supplement on Regression Computation}

\date[]{March 10, 2015}

\begin{document}

\frame{\titlepage}

\frame{\tableofcontents}

\section{Why Have This Module?}

\frame{
	\frametitle{Why a required module?}
	\begin{itemize}\itemsep1em
		\item 
		\item 
	\end{itemize}
}

\section[R]{Basic R Introduction}
\frame{\tableofcontents[currentsection]}

\frame{
	\frametitle{Why R?}
	\begin{itemize}\itemsep1em
		\item Free, Open Source Software (FOSS)
		\item Easily extensible
		\item Powerful
	\end{itemize}
}

% basic material from R course
	% multiple objects; with names
	% R object types
		% (vector, matrix, data.frame)
		% integer, numeric, logical, character, factor, etc.
	% R matrices; scan(); data editor; read.table; read.csv

\section[Matrices]{Matrix Operations}
\frame{\tableofcontents[currentsection]}

% matrix operations
	% scalar addition/subtraction/multiplication/division
	% matrix addition/subtraction
	% matrix multiplication
		% crossproduct
		% inner product
		% outer product
	% matrix transpose

% matrix activity

% OLS in matrix form

% matrix inversion
	% determinant
	% adjoint and cofactors
	% QR decomposition

% matrix inversion activity

% standard errors in matrix form


% OLS activity


\section[MLE]{Maximum Likelihood Estimation}
\frame{\tableofcontents[currentsection]}

% Likelihood functions

% Simple MLE (e.g., proportion)

	% `optim()`
	
% MLE activity

% Regression MLE


\section[Bootstrap]{Bootstrapping}
\frame{\tableofcontents[currentsection]}


% basic sampling

% observation sampling


\section[HW]{Homework Assignment}
\frame{\tableofcontents[currentsection]}

% regression in matrix form

% SEs in matrix form
% bootstrapped SEs

% logistic regression MLE
% bootstrapped SEs


\appendix
\frame{}

\end{document}
